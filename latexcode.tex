\documentclass[journal,12pt,twocolumn]{IEEEtran}

\usepackage{setspace}
\usepackage{gensymb}

\singlespacing


\usepackage[cmex10]{amsmath}

\usepackage{amsthm}

\usepackage{mathrsfs}
\usepackage{txfonts}
\usepackage{stfloats}
\usepackage{bm}
\usepackage{cite}
\usepackage{cases}
\usepackage{subfig}

\usepackage{longtable}
\usepackage{multirow}

\usepackage{enumitem}
\usepackage{mathtools}
\usepackage{steinmetz}
\usepackage{tikz}
\usepackage{circuitikz}
\usepackage{verbatim}
\usepackage{tfrupee}
\usepackage[breaklinks=true]{hyperref}
\usepackage{graphicx}
\usepackage{tkz-euclide}

\usetikzlibrary{calc,math}
\usepackage{listings}
    \usepackage{color}                                            %%
    \usepackage{array}                                            %%
    \usepackage{longtable}                                        %%
    \usepackage{calc}                                             %%
    \usepackage{multirow}                                         %%
    \usepackage{hhline}                                           %%
    \usepackage{ifthen}                                           %%
    \usepackage{lscape}     
\usepackage{multicol}
\usepackage{chngcntr}

\DeclareMathOperator*{\Res}{Res}

\renewcommand\thesection{\arabic{section}}
\renewcommand\thesubsection{\thesection.\arabic{subsection}}
\renewcommand\thesubsubsection{\thesubsection.\arabic{subsubsection}}

\renewcommand\thesectiondis{\arabic{section}}
\renewcommand\thesubsectiondis{\thesectiondis.\arabic{subsection}}
\renewcommand\thesubsubsectiondis{\thesubsectiondis.\arabic{subsubsection}}


\hyphenation{op-tical net-works semi-conduc-tor}
\def\inputGnumericTable{}                                 %%

\lstset{
%language=C,
frame=single, 
breaklines=true,
columns=fullflexible
}
\begin{document}


\newtheorem{theorem}{Theorem}[section]
\newtheorem{problem}{Problem}
\newtheorem{proposition}{Proposition}[section]
\newtheorem{lemma}{Lemma}[section]
\newtheorem{corollary}[theorem]{Corollary}
\newtheorem{example}{Example}[section]
\newtheorem{definition}[problem]{Definition}

\newcommand{\BEQA}{\begin{eqnarray}}
\newcommand{\EEQA}{\end{eqnarray}}
\newcommand{\define}{\stackrel{\triangle}{=}}
\bibliographystyle{IEEEtran}

\providecommand{\mbf}{\mathbf}
\providecommand{\pr}[1]{\ensuremath{\Pr\left(#1\right)}}
\providecommand{\qfunc}[1]{\ensuremath{Q\left(#1\right)}}
\providecommand{\sbrak}[1]{\ensuremath{{}\left[#1\right]}}
\providecommand{\lsbrak}[1]{\ensuremath{{}\left[#1\right.}}
\providecommand{\rsbrak}[1]{\ensuremath{{}\left.#1\right]}}
\providecommand{\brak}[1]{\ensuremath{\left(#1\right)}}
\providecommand{\lbrak}[1]{\ensuremath{\left(#1\right.}}
\providecommand{\rbrak}[1]{\ensuremath{\left.#1\right)}}
\providecommand{\cbrak}[1]{\ensuremath{\left\{#1\right\}}}
\providecommand{\lcbrak}[1]{\ensuremath{\left\{#1\right.}}
\providecommand{\rcbrak}[1]{\ensuremath{\left.#1\right\}}}
\theoremstyle{remark}
\newtheorem{rem}{Remark}
\newcommand{\sgn}{\mathop{\mathrm{sgn}}}
\providecommand{\abs}[1]{\left\vert#1\right\vert}
\providecommand{\res}[1]{\Res\displaylimits_{#1}} 
\providecommand{\norm}[1]{\left\lVert#1\right\rVert}
%\providecommand{\norm}[1]{\lVert#1\rVert}
\providecommand{\mtx}[1]{\mathbf{#1}}
\providecommand{\mean}[1]{E\left[ #1 \right]}
\providecommand{\fourier}{\overset{\mathcal{F}}{ \rightleftharpoons}}
%\providecommand{\hilbert}{\overset{\mathcal{H}}{ \rightleftharpoons}}
\providecommand{\system}{\overset{\mathcal{H}}{ \longleftrightarrow}}
	%\newcommand{\solution}[2]{\textbf{Solution:}{#1}}
\newcommand{\solution}{\noindent \textbf{Solution: }}
\newcommand{\cosec}{\,\text{cosec}\,}
\providecommand{\dec}[2]{\ensuremath{\overset{#1}{\underset{#2}{\gtrless}}}}
\newcommand{\myvec}[1]{\ensuremath{\begin{pmatrix}#1\end{pmatrix}}}
\newcommand{\mydet}[1]{\ensuremath{\begin{vmatrix}#1\end{vmatrix}}}

\numberwithin{equation}{subsection}

\makeatletter
\@addtoreset{figure}{problem}
\makeatother
\let\StandardTheFigure\thefigure
\let\vec\mathbf

\renewcommand{\thefigure}{\theproblem}

\def\putbox#1#2#3{\makebox[0in][l]{\makebox[#1][l]{}\raisebox{\baselineskip}[0in][0in]{\raisebox{#2}[0in][0in]{#3}}}}
     \def\rightbox#1{\makebox[0in][r]{#1}}
     \def\centbox#1{\makebox[0in]{#1}}
     \def\topbox#1{\raisebox{-\baselineskip}[0in][0in]{#1}}
     \def\midbox#1{\raisebox{-0.5\baselineskip}[0in][0in]{#1}}
\vspace{3cm}
\title{Assignment 13}
\author{KUSUMA PRIYA\\EE20MTECH11007}

\maketitle
\newpage

\bigskip
\renewcommand{\thefigure}{\theenumi}
\renewcommand{\thetable}{\theenumi}
Download codes from 
%
\begin{lstlisting}
https://github.com/KUSUMAPRIYAPULAVARTY/assignment13
\end{lstlisting}
%
 
 \section{QUESTION}
Let $\mathbbf{F}$ be a subfield of the complex numbers and let $\mathbbf{T}$ be the function from $\mathbbf{F}^3$ into $\mathbbf{F}^3$ defined by 
\begin{align}
    \mathbbf{T}(x_1,x_2,x_3)=\\(x_1-x_2+2x_3,2x_1+x_2,-x_1-2x_2+2x_3)
\end{align}
(a) Verify that $\mathbbf{T}$ is a linear transformation.\\
(b) If $(a,b,c)$ is a vector in  $\mathbbf{F}^3$, what are the conditions on $a,b,c$ that the vector be in the range of  $\mathbbf{T}$ ?What is the rank of  $\mathbbf{T}$?\\
(c) What are the conditions on $a,b,c$ that $(a,b,c)$ be in the null space of  $\mathbbf{T}$?What is the nullity of  $\mathbbf{T}$?
\end{align}

%

\section{Solution}
Representing the transformation in matrix form
\begin{align}
 \mathbbf{T}(x_1,x_2,x_3)=\vec{T}\vec{x}\\
    \vec{T}=\myvec{1&-1&2\\2&1&0\\-1&-2&2}\\
    \vec{x}=\myvec{x_1\\x_2\\x_3}
\end{align}
\subsection{Part (a)}
Consider the matrices $\vec{x},\vec{y} \in \mathbbf{F}^3$ and the scalar $c \in \mathhbbf{F}$\\
By the associativity of matrix multiplications, we can write
\begin{align}
    \vec{T}(c\vec{x}+\vec{y})=\vec{T}(c\vec{x})+\vec{T}\vec{y}\\
    =c\vec{T}\vec{x}+\vec{T}\vec{y}
\end{align}
So, $\vec{T}$ is a linear transformation.
\subsection{Part (b)}
range($\vec{T}$)=\cbrak{\vec{y}:\vec{T}\vec{x}=\vec{y} \text{ where }\vec{x},\vec{y} \in \mathbbf{F}^3}
\begin{align}
    \vec{y}=\myvec{a\\b\\c}\\
    \vec{T}\vec{x}=\vec{y}\\
    \implies \vec{B}\vec{T}\vec{x}=\vec{B}\vec{y}\\
    \implies \myvec{\frac{1}{3}&\frac{1}{3}&0\\\frac{-2}{3}&\frac{1}{3}&0\\-1&1&1}\myvec{1&-1&2\\2&1&0\\-1&-2&2}\vec{x}=\\ \myvec{\frac{1}{3}&\frac{1}{3}&0\\\frac{-2}{3}&\frac{1}{3}&0\\-1&1&1}\myvec{a\\b\\c}\\
    \myvec{1&0&\frac{2}{3}\\0&1&\frac{-4}{3}\\0&0&0}\vec{x}=\myvec{\frac{1}{3}&\frac{1}{3}&0\\\frac{-2}{3}&\frac{1}{3}&0\\-1&1&1}\myvec{a\\b\\c}\label{4}
\end{align}
So, rank($\vec{T}$)=2 and comparing the third row element in LHS and RHS of \eqref{4}
\begin{align}
  -a+b+c=0 \label{1} 
\end{align}
All vectors $\myvec{a\\b\\c} \in \mathbbf{F}^3$ that satisfy \eqref{1} lie in the range of $\vec{T}$
\subsection{Part (c)}
nullspace(\vec{T})=\cbrak{\vec{x} :\vec{T}\vec{x}=\vec{0} \text{  where  } \vec{x} \in \mathbbf{F}^3}
\begin{align}
\vec{x}=\myvec{a\\b\\c}\\
    \vec{T}\vec{x}=\vec{0}\\
    \vec{B}\vec{T}\vec{x}=\vec{0}
    \end{align}
    where $ \vec{B}\vec{T}$ is in reduced row echelon form
    \begin{align}
     \myvec{\frac{1}{3}&\frac{1}{3}&0\\\frac{-2}{3}&\frac{1}{3}&0\\-1&1&1}\myvec{1&-1&2\\2&1&0\\-1&-2&2}\vec{x}=\vec{0}\\
     \implies  \myvec{1&0&\frac{2}{3}\\0&1&\frac{-4}{3}\\0&0&0}\myvec{a\\b\\c}=\myvec{0\\0\\0}
    \\\implies a+\frac{2}{3}c=0\label{2}\\b-\frac{4}{3}c=0\label{3}
    \end{align}
    
The number of free variables in the reduced row echelon form of $\vec{T}$ is 1 hence nullity($\vec{T}$) =1\\
So, the null space of $\vec{T}$ is set of all vectors $\myvec{a\\b\\c} \in \mathbbf{F}^3$ that satisfy \eqref{2} and \eqref{3}\\
\underline{\textbf{Note}}\\
rank($\vec{T}$)+nullity($\vec{T}$)=2+1=dim($ \mathbbf{F}^3$)
\end{document}

